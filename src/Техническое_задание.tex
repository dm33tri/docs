\documentclass[a4paper, 12pt]{article}
\usepackage{./common/style}
\usepackage{graphicx}
\usepackage{placeins}
\usepackage[justification=centering]{caption}
\usepackage{float}
\usepackage[figuresleft]{rotating}

\begin{document}
	\sloppy \docNumber{RU.17701729.05.06-01 ТЗ 01-1} \docFormat{Техническое задание} \student{программы \enquote{Программная инженерия}}{Д.А. Щербаков} \project{Программа для визуализации данных о климате и погоде помощью JavaScript.}
	\supervisor{Доцент департамента программной инженерии}{Р.А. Родригес Залепинос} \firstPage
	\newpage
	\thirdPage

	\section{Введение}\label{section:1}

	\subsection{Наименование программы}\label{section:1.1}

	Наименование программы на русском языке – \enquote{Программа для визуализации данных о климате и погоде с помощью JavaScript},
	на английском языке – \enquote{Program for Visualizing Climate and Weather Data on JavaScript}.

	\subsection{Краткая характеристика области применения программы}\label{section:1.2}

	Программа для визуализации данных климата и погоды предназначена для использования в сфере метеорологии, географии и
	экологии. Она обеспечивает возможность анализа и представления динамики климатических параметров и погодных условий в удобном
	графическом виде. Область применения программы охватывает учреждения, государственные службы, а также образовательные
	организации и частные компании, занимающиеся исследованиями и мониторингом климата.

	\section{Основание для разработки}\label{section:2}

	\subsection{Документы, на основании которых ведется разработка}\label{section:2.1}

	Положение о практической подготовке студентов основных образовательных программ высшего образования – программ бакалавриата,
	специалитета и магистратуры Национального исследовательского университета \enquote{Высшая школа экономики} от 17.06.2021.

	\subsection{Наименование темы разработки}\label{section:2.2}

	Наименование темы разработки – \enquote{Программа для визуализации данных о климате и погоде с помощью JavaScript}. Условное
	обозначение разрабатываемого продукта – \enquote{Climap}.

	\subsection{Конечное основание для разработки}\label{section:2.3}

	Программа выполняется в рамках темы выпускной квалификационной работы в соответствии с учебным планом подготовки бакалавров
	по направлению 09.03.04 \enquote{Программная инженерия} Национального исследовательского университета \enquote{Высшая школа экономики},
	факультет компьютерных наук, департамент программной инженерии.

	\section{Назначение разработки}\label{section:3}

	\subsection{Функциональное назначение}\label{section:3.1}

	Основное функциональное назначение программы заключается в загрузке, обработке и визуализации данных о климате и
	погоде. Программа должна реализовывать следующие основные функции:
	\begin{itemize}
		\item Загрузка и обработка данных из различных источников, включая открытые метеорологические базы данных и
			пользовательские файлы.

		\item Визуализация данных с использованием графиков и географических карт.

		\item Фильтрация и сортировка данных по различным параметрам, таким как географическое расположение, временной диапазон
			и показатели климата.

		\item Экспорт полученных визуализаций в виде изображений.

		\item Поддержка пользовательских настроек для управления внешним видом и структурой визуализаций.
	\end{itemize}

	\subsection{Эксплуатационное назначение}\label{section:3.2}

	Программа для визуализации данных климата и погоды предназначена для использования на персональных компьютерах, ноутбуках
	и мобильных устройствах под управлением современных операционных систем. Эксплуатационное назначение программы
	включает следующие основные аспекты:
	\begin{itemize}
		\item Программа должна использоваться профессионалами и энтузиастами в области анализа данных климата и погоды.

		\item Программа должна быть доступна в виде веб-приложения, работающего в большинстве современных интернет-обозревателей
			при наличии подключения к сети \enquote{Интернет}.

		\item Программа должна поддерживать автоматическое обновление для добавления нового функционала и устранения недочетов.
	\end{itemize}

	\section{Требования к программе}\label{section:4}

	\subsection{Требования к функциональным характеристикам}\label{section:4.1}

	\subsubsection{Требования к составу выполняемых функций программы}\label{section:4.1.1}
	Программа должна выполнять следующие функции:
	\begin{itemize}
		\item Предоставлять пользователю доступ к наборам климатических и погодных данных из различных источников,
			автоматически обновлять список доступных данных.

		\item Позволять пользователю настроить визуализацию данных на карте, в виде графика или текстовых характеристик в
			зависимости от размерности и типа данных в конкретном источнике. Пользователь может выбрать тип отображения,
			цветовую палитру, а также необходимые для отображения дату и время.

		\item Отрисовывать данные в соответствующем входным параметрам виде с сохранением отзывчивости интерфейса.

		\item Использовать механизмы локального кэширования данных для сокращения количества обращений к источникам данных.

		\item Сохранять входные параметры и восстанавливать их при перезагрузке программы (страницы в интернет-обозревателе).
	\end{itemize}

	\subsubsection{Требования к организации входных данных}\label{section:4.1.2}
	Входные данные от пользователя представляют собой набор параметров, таких как:
	\begin{itemize}
		\item Источник климатических или погодных данных.

		\item Название набора данных, предоставляемый источником.

		\item Переменная из набора доступных, которую необходимо визуализировать.

		\item Дата и время среди доступных для визуализации.

		\item Тип и соответствующие параметры визуализации.
	\end{itemize}

	В качестве источников климатических данных должны быть доступны следующие базы данных:
	\begin{itemize}
		\item \textbf{ECMWF ERA5} – набор глобальных климатических данных, предоставляемый Европейским центром среднесрочных
			прогнозов погоды (ECMWF). ERA включает данные о температуре, влажности, давлении, осадках и других метеорологических
			параметрах. Доступ к ERA5 предоставляется с помощью HTTP REST API, а данные загружаются в форматах GRIB или NetCDF3.

		\item \textbf{NASA MERRA} – набор климатических данных, предоставляемый Национальным аэрокосмическим агентством США (NASA).
			MERRA включает данные о температуре, влажности, облачности, осадках, скорости ветра и других метеорологических
			параметрах. Доступ к данному источнику данных предоставляется с помощью OPeNDAP API, формат данных – NetCDF4.

		\item \textbf{NOAA GOES} – серия геостационарных спутников Национального управления океанических и атмосферных
			исследований США (NOAA), предоставляющих данные о погоде, климате и окружающей среде в режиме реального времени. Доступ
			к данным можно получить с помощью S3 API. Формат данных – NetCDF4.

		\item \textbf{NASA LAADS} – архив данных NASA, в частности данные со спутников систем MODIS Terra и Aqua. MODIS
			предоставляет данные о растительности, температуре поверхности, атмосферных явлениях, а также данные об аэрозолях и
			рассеянии света. Доступ к данным можно получить с помощью OPeNDAP API, формат данных – HDF-EOS или NetCDF.
	\end{itemize}

	\subsubsection{Требования к организации выходных данных программы}\label{section:4.1.3}
	Выходные данные представляют из себя текстовые характеристики наборов данных, графики (кривые) по текущим данным, а также
	их визуализация в виде проекции на карту мира в определённом разрешении, с определённым алгоритмом проекции (например,
	сортировка данных по квадратным участкам земли или отображение как массива точек) и соответствующими параметрами,
	такими как цветовая палитра или прозрачность слоя с данными.

	\subsubsection{Требования к временным характеристикам}\label{section:4.1.4}
	При разработке программы необходимо учитывать временную сложность алгоритмов агрегации больших данных и подобрать
	параметры для работы данных алгоритмов так, чтобы интерфейс пользователя оставался отзывчивым.

	\subsection{Требования к интерфейсу программы для пользователя}\label{section:4.2}
	Интерфейс программы должен быть разработан с учетом следующих требований:
	\begin{itemize}
		\item \textbf{Интуитивность:} интерфейс должен быть простым и интуитивно понятным для пользователей различного
			уровня подготовки.

		\item \textbf{Гибкость:} интерфейс должен предоставлять возможность настройки параметров визуализации и других
			функций программы.

		\item \textbf{Адаптивность:} интерфейс должен адаптироваться под различные устройства и размеры экрана, чтобы обеспечить
			комфортное использование программы на различных платформах.
	\end{itemize}

	Основные элементы интерфейса:
	\begin{itemize}
		\item Поля для поиска, выбора и загрузки данных климата и погоды из различных источников с различными наборами
			характеристик.

		\item Поля для настройки параметров отображения данных.

		\item Географическая карта для отображения пространственных данных с возможностью управления положением камеры и масштабом
			карты.
	\end{itemize}

	\subsection{Требования к надежности}\label{section:4.3}

	\subsubsection{Требования к обеспечению надежного (устойчивого) функционирования программы}\label{section:4.3.1}
	\begin{itemize}
		\item Программа должна обеспечивать круглосуточную доступность и непрерывность работы.

		\item Программа должна использовать отказоустойчивые компоненты и архитектуру для минимизации времени простоя.

		\item Программа должна иметь возможность автоматического восстановления после сбоев и аварий.
	\end{itemize}

	\subsubsection{Время восстановления после отказа}\label{section:4.3.2}
	При отказе программы её восстановление возможно в течение нескольких секунд вследствие перезагрузки страницы браузера.

	\subsubsection{Отказы из-за некорректных действий оператора}\label{section:4.3.3}
	Интерфейс программы не должен позволять производить действия, приводящие к некорректной работе программы. При внешнем вмешательстве,
	например, ручном редактировании базы данных браузера, программа должна распознавать некорректный ввод и устранять
	проблему, в частности сбросом состояния к начальному.

	\subsection{Условия эксплуатации}\label{section:4.4}

	\subsubsection{Климатические условия эксплуатации}\label{section:4.4.1}
	Климатические условия эксплуатации, при которых должны обеспечиваться заданные характеристики, должны удовлетворять
	требованиям, предъявляемым к техническому устройству, используемому пользователем, предоставляемые со стороны производителя
	технического устройства.

	\subsubsection{Требования к видам обслуживания}\label{section:4.4.2}
	Разработчик не оказывает обслуживание программы.

	\subsubsection{Требования к численности и квалификации персонала}\label{section:4.4.3}
	Для работы с программой не предъявляется требования к численности персонала или их квалификации.

	\subsection{Требования к составу и параметрам технических средств}\label{section:4.4.4}
	Для обеспечения работы программы необходимо наличие персонального компьютера с аппаратным обеспечением,
	соответствующим следующим характеристикам:
	\begin{itemize}
		\item Четырёх- или более ядерный процессор с максимальной тактовой частотой от 2.5 ГГц, архитектурой и техпроцессом,
			не уступающими по данным показателям семейству микропроцессоров Intel Coffee Lake.

		\item 8 Гб ОЗУ типа DDR4 или лучше.

		\item Твердотельный накопитель с объемом от 128 Гб.

		\item Устройство ввода типа \enquote{мышь} или сенсорная панель для управления указателем.

		\item Монитор или другое средство вывода изображения с разрешением не менее 1920 точек по одному, и 1080 точек по
			другому измерению.
	\end{itemize}

	\subsection{Требования к информационной и программной совместимости}\label{section:4.6}

	\subsubsection{Требования к информационным структурам и методам решения}\label{section:4.6.1}
	Требования не предъявляются.

	\subsubsection{Требования к программным средствам, используемым программой}\label{section:4.6.2}
	Для использования программы необходимо наличие программы интернет-обозревателя с поддержкой технологии WebGL,
	WebAssembly и WebWorker. Рекомендуется использование Google Chrome версии 100 и выше, Yandex Browser версии 22 и выше
	или Mozilla Firefox версии 100 и выше.

	\subsection{Требования к транспортировке и хранению}\label{section:4.7}

	\subsubsection{Требования к транспортировке и хранению программных документов, предоставленных в электронном виде}\label{section:4.7.1}
	Программные документы в электронном виде загружаются в систему LMS (Learning Management System). Требования к хранению
	и транспортировке не предъявляются.

	\subsubsection{Требования к хранению и транспортировке программных документов, предоставляемых в печатном виде}\label{section:4.7.2}
	Программные документы, предоставляемые в печатном виде, должны соответствовать общим правилам учета и хранения
	программных документов, предусмотренных стандартами ЕСПД (Единой системы программной документации) и соответствовать требованиям
	ГОСТ 19.602-78.

	\section{Требования к программной документации}\label{section:5}

	\subsection{Предварительный состав программной документации}
	\label{section:5.1}
	\begin{enumerate}
		\item \enquote{Программа для визуализации данных о климате и погоде с помощью JavaScript}. Текст программы (ГОСТ
			19.401-78 \cite{GOST_19.401-78})

		\item \enquote{Программа для визуализации данных о климате и погоде с помощью JavaScript}. Программа и методика испытаний
			(ГОСТ 19.301-79 \cite{GOST_19.301-79})

		\item \enquote{Программа для визуализации данных о климате и погоде с помощью JavaScript}. Руководство оператора (ГОСТ
			19.505-79 \cite{GOST_19.505-79})

		\item \enquote{Программа для визуализации данных о климате и погоде с помощью JavaScript}. Техническое задание (ГОСТ
			19.201-78 \cite{GOST_19.201-78})
	\end{enumerate}

	\subsection{Специальные требования к программной документации}\label{section:5.2}

	Документы к программе должны быть выполнены в соответствии с ГОСТ 19.106-78 \cite{GOST_19.106-78} и ГОСТами к каждому виду
	документа (см. п. \ref{section:5.1});

	Пояснительная записка должна быть загружена в систему Антиплагиат через LMS \enquote{НИУ ВШЭ}. Лист, подтверждающий
	загрузку пояснительной записки, сдается в учебный офис вместе со всеми материалами не позже, чем за день до защиты
	курсовой работы;

	Вся документация также воспроизводится в печатном виде, она должна быть подписана академическим руководителем образовательной
	программы 09.03.04 \enquote{Программная инженерия}, руководителем разработки и исполнителем перед сдачей курсовой
	работы в учебный офис не позже одного дня до защиты;

	Документация сдается в электронном виде в формате .pdf и .docx. проект программы в архиве формата .zip.

	Все документы перед защитой курсовой работы должны быть загружены в информационно-образовательную среду НИУ ВШЭ LMS (Learning
	management system) в личном кабинете, дисциплина – \enquote{Курсовая работа}, одним архивом.

	\section{Технико-экономические показатели}\label{section:6}

	\subsection{Предполагаемая потребность}
	Предполагается, что программа для визуализации данных климата и погоды будет востребована следующими категориями пользователей:
	\begin{itemize}
		\item Научные исследователи: ученые, занимающиеся исследованиями в области климата, метеорологии, географии и экологии.

		\item Государственные и международные организации: метеорологические службы, организации по защите окружающей среды и
			управления природными ресурсами.

		\item Образовательные учреждения: преподаватели и студенты вузов и колледжей, занимающиеся изучением климата,
			метеорологии и других смежных наук.

		\item Частные компании: предприятия, занимающиеся мониторингом и анализом климатических параметров для различных целей,
			таких как сельское хозяйство, энергетика и транспорт.

		\item Широкая аудитория: пользователи, интересующиеся климатом, погодой и окружающей средой, желающие визуализировать
			и анализировать собственные данные или данные из открытых источников.
	\end{itemize}

	\subsection{Ориентировочная экономическая эффективность}
	В рамках данной работы расчет экономической эффективности не предусмотрен.

	\subsection{Преимущества разработки по сравнению с отечественными или зарубежными аналогами}
	Основным преимуществом разработки является широкая доступность в связи с использованием распространенных протоколов и технологий,
	таких как HTTP, HTML, JavaScript и WebGL.

	\newcolumntype{L}{>{\raggedright\arraybackslash}X}
	\begin{sidewaystable}
		\begin{tabularx}{\textwidth}{|L|L|L|L|L|L|}
			\hline Сравнительная характеристика & Climate Wikience & Google Earth Engine &
			Panoply & ArcGIS & Climap \\
			\hline Разработчик & Р. А. Родригес Залепинос & Google & NASA GISS & Esri & Д. А. Щербаков \\
			\hline Цель & Анализ климатических данных & Геопространственный анализ различных данных &
			Визуализация файлов & ГИС-платформа & Визуализация данных климата и погоды \\
			\hline Форматы данных & Встроенные
			растровые, векторные и тайловые & Встроенные тайловые & NetCDF, HDF, GRIB & Растровые, векторные, текстовые & Растровые,
			векторные, текстовые \\
			\hline Визуализация & 2D и 3D карты, графики & 2D и 3D карты & 2D карты и графики & 2D и
			3D карты, графики & 2D и 3D карты, графики \\
			\hline Возможности геопространственного анализа & Обширные &
			Обширные & Ограниченные & Обширные & Обширные \\
			\hline Источники данных & Обширный набор встроенных данных (климатические
			модели, спутниковые, воздушные, векторные и другие данные) & Обширный набор источников данных & Отсутствуют & Обширный
			набор источников данных & Обширный набор источников данных с возможностью расширения \\
			\hline Лицензия & Бесплатная
			лицензия & Ограниченная бесплатная лицензия для некоммерческого использования & Бесплатная лицензия & Коммерческая
			лицензия & Бесплатная лицензия с открытым исходным кодом \\
			\hline Платформа & Приложение для ПК & Веб-приложение
			& Приложение для ПК & Приложение для ПК и веб-приложение & Веб-приложение \\
			\hline
		\end{tabularx}
		\caption{Сравнение аналогов}
	\end{sidewaystable}

	\section{Стадии и этапы разработки}\label{section:7}
	Стадии и этапы разработки были выявлены с учетом ГОСТ 19.102-77 \cite{GOST_19.102-77}:
	\begin{table}[ht]
		\begin{tabularx}
			{\textwidth}{|X|X|X|} \hline Стадии разработки & Этапы разработки & Содержание работ \\ \hline I. Техническое задание
			& \multirow{2}{=}{Обоснование необходимости разработки программы} & Постановка задачи \\ & & Сбор исходных материалов \\ & & Выбор
			и обоснование критериев эффективности и качества разрабатываемой программы \\ \cline{2-3} & Разработка и
			утверждение технического задания & Определение требований к программе \\ & & Определение стадий, этапов и сроков
			разработки программы и документации на неё \\ & & Определение необходимости проведения научно-исследовательских работ
			на последующих стадиях \\ & & Согласование и утверждение технического задания \\ \hline II. Рабочий проект &
			Разработка программы & Программирование и отладка программы \\ \cline{2-3} & Разработка документации
			& Разработка документов \\ \cline{2-3} & Испытания
			программы & Разработка, согласование и утверждение порядка и методик испытаний \\ & & Проведение предварительных
			испытаний \\ & & Корректировка программы и программной документации по результатам испытаний \\ \hline III.
			Внедрение & Подготовка и защита программного продукта & Подготовка программы и програмной документации для
			презентации и защиты \\ & & Утверждение дня защиты программы \\ & & Презентация программного продукта \\ \hline
		\end{tabularx}
	\end{table}

	\section{Порядок контроля и приёмки}\label{section:8}
	\subsection{Виды испытаний}
	Производится проверка корректного выполнения программой заложенных в неё функций, то есть осуществляется функциональное
	тестирование программы. Также осуществляется визуальная проверка интерфейса программы на соответствие пункту
	\ref{section:4.2} настоящего технического задания. Функциональное тестирование осуществляется в соответствии с документом
	\enquote{Программа для визуализации данных о климате и погоде с помощью JavaScript}. Программа и методика испытаний (ГОСТ
	19.301-79 \cite{GOST_19.301-79}), в котором указывают:
	\begin{enumerate}
		\item перечень функций программы, выделенных в программе для испытаний, и перечень требований, которым должны
			соответствовать эти функции (со ссылкой на пункт \ref{section:4.1.1} настоящего технического задания);

		\item перечень необходимой документации и требования к ней (со ссылкой на пункт \ref{section:5} настоящего
			технического задания);

		\item методы испытаний и обработки информации;

		\item технические средства и порядок проведения испытаний;
	\end{enumerate}
	Защита выполненного проекта осуществляется комиссией, состоящей из преподавателей департамента программной инженерии, в
	утверждённые приказом декана факультета компьютерных наук сроки.

	\subsection{Общие требования к приёмке работы}
	Прием программного продукта происходит при полной работоспособности программы при различных входных данных, при
	выполнении указанных в пункте \ref{section:4.1.1} настоящего документа функций, при выполнении требований указанных
	в пункте \ref{section:4.2} настоящего документа и при наличии полной документации к программе, указанной в пункте
	\ref{section:5.1}, выполненной в соответствии со специальными требованиями, указанными в пункте
	\ref{section:5.2} настоящего технического задания.

	\section{Список источников}
	\begingroup
	\renewcommand{\section}[2]{}
	\bibliographystyle{ugost2008}
	\bibliography{./common/references}
	\endgroup

	\listRegistration
\end{document}